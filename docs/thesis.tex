\documentclass[a4paper,12pt]{article}

\usepackage[T1]{fontenc}
\usepackage[utf8]{inputenc}
\usepackage[magyar]{babel}
\usepackage{amsmath,amssymb,amsthm}
\usepackage{geometry}
\geometry{margin=2.5cm}

\begin{document}

\section{Súlyozott diszkrét valószínűségi változó}

Legyen $X$ egy diszkrét valószínűségi változó, amely egy $d$ oldalú 
dobókocka kimenetelét reprezentálja. A dobókocka oldalaihoz tartozó 
kimeneteket az alábbi halmaz írja le:
\[
\{x_1, x_2, \dots, x_d\},
\]
ahol tipikusan $x_i = i$ ($1 \leq i \leq d$), de a levezetés általánosabb formában is érvényes.

Minden oldalhoz rendelünk egy nemnegatív súlyt:
\[
w_1, w_2, \dots, w_d \ge 0
\quad\text{úgy, hogy}\quad 
\sum_{i=1}^{d} w_i > 0.
\]
Ezek a súlyok határozzák meg a kimenetelek relatív esélyeit.

A súlyokból kapott elméleti valószínűségek:
\[
p_i = \frac{w_i}{\sum_{j=1}^d w_j},
\qquad i = 1,\dots,d.
\]
Ez biztosítja, hogy $0 \le p_i \le 1$ és $\sum_{i=1}^{d} p_i = 1$.

\section{Valószínűségi tömegfüggvény}

A súlyozott dobókocka eloszlását a relatív gyakoriságfüggvény (diszkrét „sűrűségfüggvény”) írja le:
\[
f_X(x_i) = P(X = x_i) = p_i,
\qquad i = 1,\dots,d.
\]

Ez a projektben az elméleti (theoretical) PMF-et jelenti, míg a szimuláció során kapott 
relatív gyakoriságok az empirikus PMF-et adják:
\[
\hat{p}_i = \frac{n_i}{n}, \qquad i = 1,\dots,d,
\]
ahol $n$ a kísérletek száma, $n_i$ pedig annak a kísérletekben előforduló számossága, hogy $X = x_i$.

\section{Eloszlásfüggvény}

A diszkrét eloszlásfüggvény definíciója:
\[
F_X(k) = P(X \le k)
       = \sum_{i :\, x_i \le k} f_X(x_i)
       = \sum_{i :\, x_i \le k} p_i.
\]

Az empirikus eloszlásfüggvény (empirical CDF) definíciója:
\[
\widehat{F}_n(k)
= \frac{1}{n} \sum_{j=1}^{n} \mathbf{1}_{\{X_j \le k\}},
\]
ahol $\mathbf{1}_{A}$ az $A$ esemény indikátorfüggvénye, azaz
\[
\mathbf{1}_{A} =
\begin{cases}
1, & \text{ha } A \text{ bekövetkezik},\\
0, & \text{különben}.
\end{cases}
\]

\section{Várható érték és második momentum}

A valószínűségi változó várható értéke:
\[
E(X) 
= \sum_{i=1}^{d} x_i\, p_i
= \sum_{i=1}^{d} x_i\, f_X(x_i).
\]

Az empirikus várható érték (mintaátlag) a relatív gyakoriságok segítségével:
\[
\widehat{E}(X)
= \sum_{i=1}^{d} x_i\, \hat{p}_i.
\]

A második momentum definíciója:
\[
E(X^2)
= \sum_{i=1}^{d} x_i^2\, p_i.
\]

\section{Variancia}

A variancia (szórásnégyzet) definíciója:
\[
D^2(X)
= E(X^2) - \bigl(E(X)\bigr)^{2}.
\]

A szimulációból származó empirikus szórásnégyzet:
\[
\widehat{D^2}(X)
= \sum_{i=1}^d x_i^2 \hat{p}_i
  - \left( \sum_{i=1}^d x_i \hat{p}_i \right)^2.
\]

\section{Hibamérés}

Az elméleti és empirikus relatív gyakoriságfüggvény különbségét az alábbi hibamértékekkel jellemezzük.

\bigskip

\textbf{Abszolút hiba:}
\[
\varepsilon_i^{(\mathrm{abs})}
= \bigl| \hat{p}_i - p_i \bigr|.
\]

\textbf{Relatív hiba:}
\[
\varepsilon_i^{(\mathrm{rel})}
= \frac{\bigl|\,\hat{p}_i - p_i\,\bigr|}{p_i},
\qquad p_i > 0.
\]

\end{document}