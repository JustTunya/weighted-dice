\documentclass[a4paper,12pt]{article}

\usepackage[T1]{fontenc}
\usepackage[utf8]{inputenc}
\usepackage[magyar]{babel}
\usepackage{amsmath,amssymb,amsthm}
\usepackage{geometry}
\geometry{margin=2.5cm}

\newtheorem{tetel}{Tétel}

\begin{document}

\section{Súlyozott diszkrét valószínűségi változó}

Legyen $X$ egy diszkrét valószínűségi változó, amely egy $d$ oldalú dobókocka
egy dobásának kimenetelét reprezentálja.
A lehetséges értékei:
\[
\{x_1,x_2,\dots,x_d\},
\]
ahol általában $x_i=i$ $(i=1,\dots,d)$.

A klasszikus, szabályos dobókocka esetén minden kimenetel azonos
valószínűségű. A továbbiakban ettől eltérő, \emph{súlyozott} esetet vizsgálunk.

Minden $x_i$ értékhez rendelünk egy nemnegatív súlyt:
\[
w_1,w_2,\dots,w_d \ge 0,
\qquad
\sum_{i=1}^{d} w_i > 0.
\]

A súlyokból a valószínűségeket normalizálással definiáljuk:
\[
p_i = \frac{w_i}{\sum_{j=1}^{d} w_j},
\qquad i=1,\dots,d.
\]

Ekkor $0 \le p_i \le 1$ és
\[
\sum_{i=1}^{d} p_i = 1,
\]
tehát $\{p_i\}$ valóban egy diszkrét eloszlás paraméterei.

\section{Súlyok meghatározása geometriai paraméterek alapján}

A továbbiakban a $d=6$ esetet tekintjük.
Feltesszük, hogy a dobókocka nem feltétlenül szabályos kocka, hanem egy
téglatest alakú test, amelynek élei:
\[
l_x>0, \qquad l_y>0, \qquad l_z>0.
\]

A dobókockának három pár egymással szemközti lapja van.
Ezek felületei:
\[
A_{xz} = l_x l_z \quad (1.\text{ és }6.\text{ lap}),
\]
\[
A_{yz} = l_y l_z \quad (2.\text{ és }5.\text{ lap}),
\]
\[
A_{xy} = l_x l_y \quad (3.\text{ és }4.\text{ lap}).
\]

\medskip
A geometriai különbségek hatását a dobások kimenetelére
valószínűségi módon modellezzük.
A modell alapfeltevése szerint a nagyobb felületű lapok
nagyobb súlyt kapnak.

Legyen $\gamma>0$ egy alakparaméter.
A súlyokat az alábbi módon definiáljuk:
\[
w_1 = w_6 = (A_{xz})^{\gamma}, \qquad
w_2 = w_5 = (A_{yz})^{\gamma}, \qquad
w_3 = w_4 = (A_{xy})^{\gamma}.
\]

A hozzájuk tartozó valószínűségeket ismét normalizálással kapjuk:
\[
p_i = \frac{w_i}{\sum_{j=1}^{6} w_j}, \qquad i=1,\dots,6.
\]

\medskip
Megjegyzés:
\begin{itemize}
  \item $\gamma=1$ esetén a súlyok arányosak a felületekkel.
  \item $\gamma>1$ esetén a különbségek felerősödnek.
  \item $0<\gamma<1$ esetén a különbségek mérséklődnek.
\end{itemize}

\section{Belső üreg hatása – valószínűségi modell}

Tegyük fel, hogy a dobókocka anyaga homogén, de belsejében egy gömb alakú
üreg található.
Ez az üreg a tömegközéppontot eltolja, ami a dobások eloszlását
torzíthatja.

Legyen az üreg sugara $r \ge 0$, középpontjának eltolása pedig
$\mathbf b = (b_x,b_y,b_z)$.

A tömegközéppont eltolódását a következő vektorral modellezzük:
\[
\mathbf\Delta
=
-\frac{V_{\mathrm{bub}}}{V_{\mathrm{solid}}}\,\mathbf b.
\]

A modellben a súlyokat egy korrekciós tényezővel módosítjuk:
\[
\widetilde w_i
=
w_i \exp\!\bigl(-k \langle \mathbf\Delta, \mathbf n_i\rangle\bigr),
\qquad i=1,\dots,6,
\]
ahol $\mathbf n_i$ az $i$-edik lap kifelé mutató egységnormálvektora,
$k>0$ pedig egy hangolható paraméter.

Az új elméleti valószínűségek:
\[
\widetilde p_i
=
\frac{\widetilde w_i}{\sum_{j=1}^{6} \widetilde w_j}.
\]

Ha nincs üreg ($r=0$), akkor visszakapjuk az eredeti modellt.

\section{Relatív gyakoriságfüggvény (PMF)}

A súlyozott dobókocka eloszlását a relatív gyakoriságfüggvény írja le:
\[
f_X(x_i) = P(X=x_i) = p_i,
\qquad i=1,\dots,d.
\]

\section{Mintavétel és empirikus eloszlás}

Végezzünk $n\ge1$ darab független dobást:
\[
X_1,X_2,\dots,X_n.
\]

Jelölje
\[
n_i = \sum_{j=1}^{n} \mathbf 1_{\{X_j=x_i\}}
\]
az $x_i$ érték előfordulásainak számát.

Az empirikus relatív gyakoriság:
\[
\hat p_i = \frac{n_i}{n},
\qquad i=1,\dots,d.
\]

\section{Eloszlásfüggvény (CDF)}

Az elméleti eloszlásfüggvény:
\[
F_X(k) = P(X\le k) = \sum_{i:\,x_i\le k} p_i.
\]

Az empirikus eloszlásfüggvény:
\[
\widehat F_n(k)
=
\frac{1}{n}\sum_{j=1}^{n} \mathbf 1_{\{X_j\le k\}}.
\]

\section{Inversion by Sequential Search}

A nemegyenletes diszkrét mintavételezés az inverziós módszer
szekvenciális kereséssel történik.

Definiáljuk a kumulatív összegeket:
\[
F_i = \sum_{k=1}^{i} p_k,
\qquad i=1,\dots,d,
\]
ahol $F_d=1$.

Generáljunk egy $U\sim U(0,1)$ egyenletes eloszlású változót, majd:
\[
I = \min\{i:\, F_i \ge U\},
\qquad X=x_I.
\]

Ez biztosítja, hogy $P(X=x_i)=p_i$ teljesüljön.

\section{Várható érték és második momentum}

Az elméleti várható érték:
\[
E(X) = \sum_{i=1}^{d} x_i p_i.
\]

A mintabeli várható érték:
\[
\overline X_n = \frac{1}{n}\sum_{j=1}^{n} X_j
= \sum_{i=1}^{d} x_i \hat p_i.
\]

A második nyers momentum:
\[
E(X^2)=\sum_{i=1}^{d} x_i^2 p_i,
\qquad
\widehat E(X^2)=\sum_{i=1}^{d} x_i^2 \hat p_i.
\]

\section{Variancia}

Az elméleti variancia:
\[
D^2(X) = E(X^2) - (E(X))^2.
\]

Az empirikus becslés:
\[
\widehat{D^2}(X)
=
\sum_{i=1}^{d} x_i^2 \hat p_i
-
\left(\sum_{i=1}^{d} x_i \hat p_i\right)^2.
\]

\section{Az empirikus átlag konvergenciája}

\begin{tetel}[Gyenge nagy számok törvénye]
Ha $X_1,X_2,\dots$ független, azonos eloszlású változók és $E(X_1)$ létezik, akkor
\[
\overline X_n \xrightarrow[n\to\infty]{} E(X_1)
\]
valószínűségben.
\end{tetel}

\section{Hibamérés}

Az empirikus és elméleti valószínűségek eltérését mérhetjük:

\textbf{Abszolút hiba:}
\[
\varepsilon_i^{(\mathrm{abs})} = |\hat p_i - p_i|.
\]

\textbf{Relatív hiba:}
\[
\varepsilon_i^{(\mathrm{rel})}
=
\frac{|\hat p_i - p_i|}{p_i},
\qquad p_i>0.
\]

\end{document}